\left(\frac{1}{2}\right){u_2}_{(0)}{z}_{(0)} + {z}_{(2)}